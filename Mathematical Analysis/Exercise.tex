\documentclass[12pt]{report}
\usepackage[T1]{fontenc}
\usepackage[utf8]{inputenc}
\usepackage{lmodern}

\usepackage{hyperref}
\usepackage{graphicx}
\usepackage[english]{babel}

\usepackage{amsthm}
\usepackage{amssymb}
\usepackage{amsmath}
\usepackage{bbm}

\usepackage{enumitem}

\newcommand{\sol}{{\textbf{Solution~}}}
\newcommand{\R}{{\mathbf{R}}}

%use \section but to change secnumdepth so that they don't get numbered.
\setcounter{secnumdepth}{0}

% Book's title and subtitle
\title{\Huge \textbf{Understanding Analysis Practice} %\footnote{This is a footnote.} 
\\ \huge Solutions for some exercise problems %\footnote{This is yet another footnote.}
}
% Author
\author{\textsc{Yaze Li}%\thanks{\url{www.example.com}}
}

\begin{document}
%\frontmatter
\maketitle
\tableofcontents
%\mainmatter
\chapter{The Real Numbers}
\section{1.2 ~~Some Preliminaries}
\subsection*{Exercise 1.2.9}
Given a function $f:D \to \mathbf{R} $ and a subset $B \subseteq  \mathbf{R}$, let $f^{-1}(B)$ be the set of all points from the domain $D$ that get mapped into $B$;
that is, $f^{-1}(B) = \{x \in D: f(x) \in B\}$. This set is called the \textit{preimage} of $B$.
\begin{enumerate}[label=(\alph*)]
    \item Let $f(x)=x^2$. If $A$ is the closed interval $[0,4]$ and $B$ is the closed interval $[-1,1]$, find $f^{-1}(A)$ and $f^{-1}(B)$. Does $f^{-1}(A \cap B) = f^{-1}(A) \cap f^{-1}(B)$ in this case?
    Does $f^{-1}(A \cup B) = f^{-1}(A) \cup f^{-1}(B)$?\\
    \textbf{Solution} $f^{-1}(A)=[-2,2],f^{-1}(B)=[-1,1]$.\\
    $f^{-1}(A \cap B) = f^{-1}([0,1]) = [-1,1] = f^{-1}(A) \cap f^{-1}(B)$.\\
    $f^{-1}(A \cup B) = f^{-1}([-1,4]) = [-2,2] = f^{-1}(A) \cup f^{-1}(B)$.
    \item The good behavior of preimages demonstrated in (a) is completely general. Show that for an arbitrary function $g: \mathbf{R} \rightarrow \mathbf{R}$, it is always true that $g^{-1}(A \cap B)=g^{-1}(A) \cap g^{-1}(B)$ and $g^{-1}(A \cup B)=g^{-1}(A) \cup g^{-1}(B)$ for all sets $A, B \subseteq \mathbf{R}$.\\
    \textbf{Solution} We show that $g^{-1}(A \cap B)=g^{-1}(A) \cap g^{-1}(B)$.\\
    $(\Rightarrow)$ If $x \in g^{-1}(A \cap B)$, then by definition $g(x) \in A \cap B$. By the definition of intersection, $g(x) \in A$ and $g(x) \in B$.
    Then $x \in g^{-1}(A)$ and $x \in g^{-1}(B)$, $x \in g^{-1}(A) \cap g^{-1}(B)$.\\
    $(\Leftarrow)$ Apply the abov procedure backward.\\
    The proof of $g^{-1}(A \cup B)=g^{-1}(A) \cup g^{-1}(B)$ is similar, just change the intersection into union, and into or.
\end{enumerate}

\section{1.3 ~~The Axiom of Completeness}
\subsection*{Exercise 1.3.1}
\begin{enumerate}[label=(\alph*)]
    \item Write a formal definition in the style of Definition 1.3.2 for the \textit{infimum} or \textit{greatest lower bound} of a set.\\
    \textbf{Solution} A real number $i$ is the \textit{greatest lower bound} of a set $A \subseteq \mathbf{R}$ if it meets the following two criteria:
    \begin{enumerate}[label=(\roman*)]
        \item $i$ is a lower bound of $A$;
        \item if $b$ is any lower bound for $A$, then $i \geq b$.
    \end{enumerate}
    \item Now, state and prove a version of Lemma 1.3.8 for greatest lower bounds.\\
    \textbf{Solution} State: \textit{Assume $i \in \mathbf{R}$ is a lower bound for a set $A\subseteq\mathbf{R}$. Then, $i = \inf A$ if and only if , for every choice of $\epsilon >0$, there exists an element $a \in A$ satisfying $i+\epsilon >a$.}\\
    \textit{Proof.}\\
    $(\Rightarrow)$ Assume $i = \inf A$, and $i+\epsilon >i$. $i+\epsilon$ is not a lower bound of $A$ based on the criteria (ii) above. Then there must be some element $a \in A$ such that $i+\epsilon >a$.\\
    $(\Leftarrow)$ Assume $i$ is a lower bound such that for all $\epsilon>0$, $i+\epsilon$ is not a lower bound of $A$. Let $b = i +\epsilon$, then this implies that if $i<b$, then $b$ is not a lower bound of $A$. This is the contrapositive statement of the criteria (ii) above.\qed
\end{enumerate}

\subsection*{Exercise 1.3.6}
Given sets $A$ and $B$, define $A+B=\{a+b: a \in A$ and $b \in B\}$. Follow these steps to prove that if $A$ and $B$ are nonempty and bounded above then $\sup (A+B)=\sup A+\sup B$.
\begin{enumerate}[label=(\alph*)]
    \item Let $s=\sup A$ and $t=\sup B$. Show $s+t$ is an upper bound for $A+B$.
    \textbf{Solution} Since $s=\sup A$ and $t=\sup B$, $\forall (a \in A, b \in B), s \geq a, t \geq b$. Then $\forall a+b \in A+B, s+t \geq a+b$, $s+t$ is an upper bound for $A+B$.
    \item Now let $u$ be an arbitrary upper bound for $A+B$, and temporarily fix $a \in A$. Show $t \leq u-a$.\\
    \textbf{Solution} $\forall a+b \in A+B, u \geq a+b, u-a \geq b$. Fix $a\in A$, $u-a$ is a upper bound of $B$. Since $t=\sup B$, $t \leq u-a$.
    \item Finally, show $\sup (A+B)=s+t$.\\
    \textbf{Solution} (a) has shown that $s+t$ is an upper bound for $A+B$, we only need to show that it's the least. That is if $u$ be an arbitrary upper bound for $A+B$, then $s+t \leq u$.\\
    From (b) we have $\forall a \in A, t\leq u-a$. $s = \sup A$, then $\forall a\in A, s \geq a$, $t\leq u-a\leq u-s$. Thus $s+t \leq u$.
    \item Construct another proof of this same fact using Lemma 1.3.8.\\
    \textbf{Solution} We need to show that for every choice of $\epsilon >0$, there exists an element $a+b \in A+B$ satisfying $s+t-\epsilon <a+b$.\\
    Given the choice of $\epsilon >0$, since $s$ and $t$ are least upper bounds, we apply Lemma 1.3.8 to them:\\
    $\forall \epsilon_1 >0, \epsilon_2 >0, \exists a \in A, b \in B$, s.t. $s-\epsilon_1 < a, t-\epsilon_2 < b$. Taking $\epsilon_1 = \epsilon_2 = \frac{\epsilon}{2}$ and adding two inequalities finish the proof.
\end{enumerate}

\subsection*{Exercise 1.3.11}
Decide if the following statements about suprema and infima are true or false. Give a short proof for those that are true. For any that are false, supply an example where the claim in question does not appear to hold.
\begin{enumerate}[label=(\alph*)]
    \item If $A$ and $B$ are nonempty, bounded, and satisfy $A \subseteq B$, then $\sup A \leq$ $\sup B$.\\
    \textbf{Solution} True. $\forall a \in A, a \leq \sup A$, $\forall b \in B, b \leq \sup B$. Since $A \subseteq B$, $\forall a \in A, a \in B$. Then $\forall a \in A, a \leq \sup B$, $\sup B$ is an upper bound for $A$. By the definition of $\sup A$, $\sup A \leq \sup B$.
    \item If $\sup A<\inf B$ for sets $A$ and $B$, then there exists a $c \in \mathbf{R}$ satisfying $a<c<b$ for all $a \in A$ and $b \in B$.\\
    \textbf{Solution} True. $\forall c \in (\sup A,\inf B), (\forall a \in A, b \in B, a<c<b)$. 
    \item If there exists a $c \in \mathbf{R}$ satisfying $a<c<b$ for all $a \in A$ and $b \in B$, then $\sup A<\inf B$.\\
    \textbf{Solution} False. If $c = \sup A =\inf B$, then $\forall a \in A, b \in B, a<c<b$, and $\sup A = \inf B$.
\end{enumerate}

\section{1.4 ~~Consequences of Completeness}
\subsection*{Exercise 1.4.2}
Let $A \subseteq \mathbf{R}$ be nonempty and bounded above, and let $s \in \mathbf{R}$ have the property that for all $n \in \mathbf{N}, s+\frac{1}{n}$ is an upper bound for $A$ and $s-\frac{1}{n}$ is not an upper bound for $A$. Show $s=\sup A$.\\
\textbf{Solution} Proof by contradiction.\\
If $s > \sup A$, then let $\epsilon = s-\sup A > 0$, there must exist $n \in \mathbf{N}$ s.t. $n > \frac{1}{\epsilon}$. $s-\frac{1}{n} > s- \epsilon = \sup A$. Then $s-\frac{1}{n}$ is an upper bound of $A$, contradiction.
If $s < \sup A$, then similarly, there exists $n$ contradicts to $s+\frac{1}{n}$ being an upper bound. \qed

\subsection*{Exercise 1.4.8}
Give an example of each or state that the request is impossible. When a request is impossible, provide a compelling argument for why this is the case.
\begin{enumerate}[label=(\alph*)]
    \item Two sets $A$ and $B$ with $A \cap B=\emptyset, \sup A=\sup B, \sup A \notin A$ and $\sup B \notin B$.\\
    \textbf{Solution} Let $A = \{1-\frac{1}{2n}: n \in \mathbf{N}\}, B = \{1-\frac{1}{2n-1}: n \in \mathbf{N}\}$, then $A\cap B = \emptyset$. $\sup A = \sup B = 1, 1 \notin A$ and $1 \notin B$. 
    \item A sequence of nested open intervals $J_1 \supseteq J_2 \supseteq J_3 \supseteq \cdots$ with $\bigcap_{n=1}^{\infty} J_n$ nonempty but containing only a finite number of elements.\\
    \textbf{Solution} Let $J_n = (-\frac{1}{n},\frac{1}{n}), n\in \mathbf{N}$, then $J_1 \supseteq J_2 \supseteq J_3 \supseteq \cdots$ with $\bigcap_{n=1}^{\infty} J_n = \{0\}$.
    \item A sequence of nested unbounded closed intervals $L_1 \supseteq L_2 \supseteq L_3 \supseteq \cdots$ with $\bigcap_{n=1}^{\infty} L_n=\emptyset$. (An unbounded closed interval has the form $[a, \infty)=$ $\{x \in R: x \geq a\}$.)\\
    \textbf{Solution} Let $L_n = [n,\infty), n\in \mathbf{N}$, then $L_1 \supseteq L_2 \supseteq L_3 \supseteq \cdots$ with $\bigcap_{n=1}^{\infty} L_n=\emptyset$.
    \item A sequence of closed bounded (not necessarily nested) intervals $I_1, I_2$, $I_3, \ldots$ with the property that $\bigcap_{n=1}^N I_n \neq \emptyset$ for all $N \in \mathbf{N}$, but $\bigcap_{n=1}^{\infty} I_n=\emptyset$.\\
    \textbf{Solution} Impossible. Let $K_n = \bigcap_{m=1}^n I_m$, then for each $n \in \mathbf{N}$, $K_n$ is a closed interval and $K_n \subseteq K_{n+1}$ since $K_{n+1} = K_n \cap I_{n+1}$.\\
    By the \textbf{Nested Interval Property} (Theorem 1.4.1), $\bigcap_{n=1}^{\infty} K_n \neq \emptyset$. Well, $\bigcap_{n=1}^{\infty} K_n = \bigcap_{n=1}^{\infty} (\bigcap_{m=1}^n I_m) = \bigcap_{n=1}^{\infty} I_n \neq \emptyset$.
\end{enumerate}

\chapter{Sequences and Series}
\section{2.2 ~~The Limit of a Sequence}
\subsection*{Exercise 2.2.2}
Verify, using the definition of convergence of a sequence, that the following sequences converge to the proposed limit.
\begin{enumerate}[label=(\alph*)]
    \item $\lim \frac{2n+1}{5n+4}=\frac{2}{5}$.\\
    \textbf{Solution} \textit{Proof} For any $n \in \mathbf{N}$,
    \begin{align*}
        \bigg| \frac{2n+1}{5n+4}-\frac{2}{5}\bigg| = \bigg|\frac{5(2n+1)-2(5n+4)}{5(5n+4)}\bigg| = \bigg|-\frac{3}{5(5n+4)}\bigg|=\frac{3}{25n+20}<\frac{25}{25n}=\frac{1}{n}
    \end{align*}
    so any integer $N \geq \frac{1}{\epsilon}$ will satisfy the definition. \qed
    \item $\lim \frac{2 n^2}{n^3+3}=0$.\\
    \textbf{Solution} \textit{Proof} For any $n \in \mathbf{N}$,
    \begin{align*}
        \bigg| \frac{2n^2}{n^3+3}-0\bigg| = \frac{2n^2}{n^3+3} <\frac{2n^2}{n^3}=\frac{2}{n}
    \end{align*}
    so any integer $N \geq \frac{2}{\epsilon}$ will satisfy the definition. \qed
    \item $\lim \frac{\sin \left(n^2\right)}{\sqrt[3]{n}}=0$.\\
    \textbf{Solution} \textit{Proof} For any $n \in \mathbf{N}$,
    \begin{align*}
        \bigg| \frac{\sin \left(n^2\right)}{\sqrt[3]{n}}-0\bigg| = \frac{\sin \left(n^2\right)}{\sqrt[3]{n}} \leq\frac{1}{\sqrt[3]{n}}
    \end{align*}
    so any integer $N \geq \frac{1}{\epsilon^3}$ will satisfy the definition. \qed
\end{enumerate}
\subsection*{Exercise 2.2.4}
Give an example of each or state that the request is impossible. For any that are impossible, give a compelling argument for why that is the case.
\begin{enumerate}[label=(\alph*)]
    \item A sequence with an infinite number of ones that does not converge to one.\\
    \textbf{Solution} $x_n = (-1)^n$ has an infinite number of ones but diverges.
    \item A sequence with an infinite number of ones that converges to a limit not equal to one.\\
    \textbf{Solution} Impossible. First, an infinite number of ones means that $\forall N \in \mathbf{N}, \exists n \geq N$, s.t. $a_n = 1$. Otherwise there will be finite ones.\\
    Suppose $\lim a_n = a \neq 1$, take $\epsilon = |a-1|$. Then $\forall N \in \mathbf{N}, \exists n \geq N$, s.t. $a_n = 1, |a_n-a|=|1-a|=\epsilon$, violates the definition of limit.
    \item A divergent sequence such that for every $n \in \mathbf{N}$ it is possible to find $n$ consecutive ones somewhere in the sequence.\\
    \textbf{Solution} $x_n = (1,-1,1,1,-1,1,1,1,-1,\cdots)$
\end{enumerate}

\subsection*{Exercise 2.2.6}
Prove Theorem 2.2.7. To get started, assume $\left(a_n\right) \rightarrow a$ and also that $\left(a_n\right) \rightarrow b$. Now argue $a=b$.\\
\textbf{Solution} \textit{Proof} Let any $\epsilon > 0 $ be given. Define $\epsilon_1 = \epsilon_2 = \frac{\epsilon}{2} >0$.\\
Since $(a_n) \to a$, $\exists N_1 \in \mathbf{N}$, s.t. $\forall n \geq N_1, |a_n-a|<\epsilon_1$.\\
Similarly, $\exists N_2 \in \mathbf{N}$, s.t. $\forall n \geq N_2, |a_n-b|<\epsilon_2$.\\
Let $N = \max\{N_1,N_2\}$, for any $n>N$, apply the triangle inequality:
\begin{align*}
    |a-b| \leq |a-a_n|+|a_n-b| < \epsilon_1+\epsilon_2 = \epsilon
\end{align*}
This proves, $\forall \epsilon >0 , |a-b| < \epsilon$, which implies $|a-b|=0, a=b$. \qed

\section{2.3 ~~The Algebraic and Order Limit Theorems}
\subsection*{Exercise 2.3.3 (Squeeze Theorem)}
Show that if $x_n \leq y_n \leq z_n$ for all $n \in \mathbf{N}$, and if $\lim x_n=\lim z_n=l$, then $\lim y_n=l$ as well.\\
\textbf{Solution} \textit{Proof} Let $\epsilon >0$ be given. Since $\lim x_n = l$, $\exists N_1 \in \mathbf{N}$, s.t. $\forall n \geq N_1, |x_n-l| < \epsilon$.\\
Rewrite it as:
\begin{align*}
    \exists N_1 \in \mathbf{N}, \forall n > N_1, l-\epsilon< x_n < l+\epsilon.
\end{align*}
Similarly,
\begin{align*}
    \exists N_2 \in \mathbf{N}, \forall n > N_2, l-\epsilon< z_n < l+\epsilon.
\end{align*}
Let $N = \max\{N_1,N_2\}$, for any $n>N$, apply $x_n\leq y_n \leq z_n$:
\begin{align*}
    l-\epsilon <x_n \leq y_n \leq z_n < l+\epsilon, ~~~\text{so}~~~ |y_n-l|<\epsilon.
\end{align*}
This proves, $\forall \epsilon >0 , \exists N \in \mathbf{N} $, s.t.$\forall n > N, |y_n-l| < \epsilon$, which implies $\lim y_n = l$. \qed

\subsection*{Exercise 2.3.10}
Consider the following list of conjectures. Provide a short proof for those that are true and a counterexample for any that are false.
\begin{enumerate}[label=(\alph*)]
    \item If $\lim(a_n-b_n)=0$, then $\lim a_n = \lim b_n$.\\
    \textbf{Solution} False. Consider $a_n = b_n = n$, then $\lim (a_n-b_n)=0$ and $\lim a_n, \lim b_n$ DNE.
    \item If $(b_n)\to b$, then $|b_n|\to |b|$.\\
    \textbf{Solution} True. If $(b_n)\to b$, then $\forall \epsilon >0, \exists N \in \mathbf{N}, \forall n \geq N, |b_n-b| < \epsilon$.\\
    Under the same $\epsilon,N$, $\forall n \geq N, \big| |b_n|-|b| \big| \leq |b_n-b|<\epsilon$. Thus, $|b_n|\to |b|$.
    \item If $(a_n) \to a$ and $(b_n-a_n) \to 0$, then $(b_n)\to a$.\\
    \textbf{Solution} True. Apply the \textbf{Algebraic Limit Theorem} (ii), $\lim b_n = \lim [a_n + (b_n-a_n)] = \lim a_n + \lim (b_n-a_n) = a+0 = a$.
    \item If $(a_n) \to 0$ and $|b_n-b| \leq a_n$ for all $n \in \mathbf{N}$, then $(b_n)\to b$.\\
    \textbf{Solution} True. If $(a_n)\to 0$, then $\forall \epsilon >0, \exists N \in \mathbf{N}, \forall n \geq N, |a_n| < \epsilon$.\\
    Under the same $\epsilon,N$, $\forall n \geq N, |b_n-b|\leq a_n \leq |a_n| <\epsilon$. Thus, $(b_n)\to b$.
\end{enumerate}

\section{2.4 ~~The Monotone Convergence Theorem and a First Look at Infinite Series}
\subsection*{Practice question}
Let $(a_n)$ be a sequence such that $|a_{n+1}-a_n|<\frac{1}{2^n}$ for all $n \in \mathbf{N}$. Prove that $(a_n)$ is a convergent sequence.\\
(\textbf{Hint}: Show that $(a_n)$ is a Cauchy sequence.)\\
\textbf{Solution} \textit{Proof} Given $\epsilon > 0$, choose $N \in \mathbf{N}$ such that $\frac{1}{2^{N-1}}< \epsilon$. Then for any $n \geq N$ and $p \in \mathbf{N}$,
\begin{align*}
    |a_{n+p} -a_n| & \leq |a_n -a_{n+1}| + |a_{n+1}-a_{n+2}| + \ldots + |a_{n+p-1}-a_{n+p}|\\
                    & \leq \frac{1}{2^n} +\frac{1}{2^{n+1}} +\ldots + \frac{1}{2^{n+p}}\\
                    & \leq \frac{1}{2^n}\big[ 1+ \frac{1}{2} +\frac{1}{2^2} +\ldots+\frac{1}{2^p}\big]\\
                    & < \frac{2}{2^n} = \frac{1}{2^{n-1}} < \epsilon 
\end{align*}
So $(a_n)$ is a Cauchy sequence, and by \textbf{Theorem 2.6.4 (Cauchy Criterion)}, it converges.

\subsection*{Exercise 2.4.3}
\begin{enumerate}[label=(\alph*)]
    \item Show that
    \begin{align*}
        \sqrt{2},\sqrt{2+\sqrt{2}},\sqrt{2+\sqrt{2+\sqrt{2}}},\ldots
    \end{align*}
    converges and find the limit.\\
    \textbf{Solution} $x_1 = \sqrt2$,
    \begin{align*}
         x_{n+1} = \sqrt{2+x_n},
    \end{align*}
    Using induction, $n=1: x_2 = \sqrt{2+\sqrt{2}} >x_1, x_1 <2$, suppose $n=i-1: x_i > x_{i-1}, x_{i-1}<2$, then:
    \begin{align*}
        2+x_i > 2+x_{i-1} &\Rightarrow \sqrt{2+x_i} > \sqrt{2+x_{i-1}} \Rightarrow x_{i+1} > x_i\\
        x_i &= \sqrt{2+x_{i-1}} < \sqrt{2+2} = 2
    \end{align*}
    Thus $x_n$ is increasing and bounded $\sqrt2 \leq x_n <2$, by \textbf{Theorem 2.4.2 (Monotone Convergence Theorem)} it converges.\\
    Let $x = \lim x_n$, by taking limit of each side of the recursive equation in part (a):
    \begin{align*}
        x = \sqrt{2+x}
    \end{align*}
    we have $x=-1$ or $x=2$, since $x>\sqrt2$, $x=2$.
    \item Does the sequences
    \begin{align*}
        \sqrt{2},\sqrt{2\sqrt{2}},\sqrt{2\sqrt{2\sqrt{2}}},\ldots
    \end{align*}
    converges? If so, find the limit.\\
    \textbf{Solution} $y_1 = \sqrt2$,
    \begin{align*}
         y_{n+1} = \sqrt{2y_n},
    \end{align*}
    Using induction, $n=1: y_2 = \sqrt{2\sqrt{2}} >y_1, y_1 <2$, suppose $n=i-1: y_i > y_{i-1}, y_{i-1}<2$, then:
    \begin{align*}
        2y_i > 2y_{i-1} &\Rightarrow \sqrt{2y_i} > \sqrt{2y_{i-1}} \Rightarrow y_{i+1} > y_i\\
        y_i &= \sqrt{2y_{i-1}} < \sqrt{2\times2} = 2
    \end{align*}
    Thus $y_n$ is increasing and bounded $\sqrt2 \leq y_n <2$, by \textbf{Theorem 2.4.2 (Monotone Convergence Theorem)} it converges.\\
    Let $y = \lim y_n$, by taking limit of each side of the recursive equation in part (a):
    \begin{align*}
        y = \sqrt{2y}
    \end{align*}
    we have $y=0$ or $y=2$, since $y>\sqrt2$, $y=2$.
\end{enumerate}

\section{2.5 ~~Subsequences and the Bolzano–Weierstrass Theorem}
\subsection*{Exercise 2.5.2}
Decide whether the following propositions are true or false, providing a short justification for each conclusion.
\begin{enumerate}[label=(\alph*)]
    \item If every proper subsequence of $(x_n)$ converges, then $(x_n)$ converges as well.\\
    \textbf{Solution} True. Sequence $(x_2,x_3,x_4,\ldots)$ is a proper subsequence of $(x_n)$, so it converges. Then $(x_n)$ converges since the first 
    term does not change the convergence of a sequence.
    \item If $(x_n)$ contains a divergent subsequence, then $(x_n)$ diverges.\\
    \textbf{Solution} True. \textbf{Theorem 2.5.2} shows that subsequence of a convergent sequence converges, which is the contrapositive of (b).
    \item If $(x_n)$ is bounded and diverges, then there exist two subsequences of $(x_n)$
    that converge to different limits.\\
    \textbf{Solution} True. 
    \iffalse{
        Define:
    \begin{align*}
        s_n &= \sup\{x_n: k\geq n\},\\
        t_n &= \inf\{x_n: k\geq n\}
    \end{align*}
    $(s_n)$ is bounded and decreasing, by \textbf{Theorem 2.5.2 (Monotone Convergence Theorem)}, $(s_n)$ converges. So does $(t_n)$.\\
    The limit superior (upper limit) and limit inferior (lower limit) are defined as:
    \begin{align*}
        \lim \sup x_n = \lim s_n, \text{~~~and~~~}\lim \inf x_n = \lim t_n
    \end{align*}
    We first show that there exists a subsequence $(x_{n_k})$ of $(x_n)$ that converges to $\lim \sup x_n$.\\
    Suppose $\lim \sup x_n = \lim s_n = \ell$, i.e. for any $\epsilon >0$, 
    there exists $N \in \mathbf{N}$ such that
    \begin{align*}
        \forall n \geq N, \ell-\epsilon<s_n<\ell+\epsilon
    \end{align*}
    Thus $s_N = \sup\{x_n: n\geq N\}<\ell+\epsilon$, and $\forall n \geq N, x_n<\ell+\epsilon$.\\
    For $\epsilon=1$, there exists $N_1 \in\mathbf{N}$ such that
    \begin{align*}
        \ell-1 < s_{N_1} = \sup\{x_n:n \geq N_1\}<\ell+1
    \end{align*}
    Thus, $\exists n_1\in \mathbf{N}, \ell-1<a_{n_1}<\ell+1$. Similarly for $\epsilon = \frac{1}{k}$,
    there exists $N_k \in\mathbf{N}$ such that
    \begin{align*}
        \ell-\frac{1}{k} < s_{N_k} = \sup\{x_n:n \geq N_k\}<\ell+\frac{1}{k}
    \end{align*}
    and $\exists n_k\in \mathbf{N}, \ell-\frac{1}{k}<a_{n_k}<\ell+\frac{1}{k}$.\\
    Therefore, $\lim x_{n_k} = \ell$. Similarly, there exists a subsequence of $(x_n)$ that converges to $\lim \inf x_n$.
    Next we show that if $(x_n)$ diverges,  $\lim \sup x_n \neq \lim \inf x_n$.\\
    Suppose $\lim \sup x_n = \lim \inf x_n = \ell$, by definition:
    \begin{align*}
        &\forall \gamma < \ell, \exists N_1 \in \mathbf{N}: \forall n \geq N_1, x_n > \gamma,\\
        &\forall \lambda >\ell, \exists N_2 \in \mathbf{N}: \forall n \geq N_2, x_n < \lambda. 
    \end{align*}
    Let $N = \max\{N_1,N_2\}$, given any $\epsilon>0$, choose $\gamma = \ell-\epsilon, \lambda=\ell+\epsilon$, then
    \begin{align*}
        \forall n>N, \text{~~~~} \ell-\epsilon<x_n<\ell+\epsilon
    \end{align*}
    $(x_n)\to \ell$, contradicts to $(x_n)$ diverges.\\
    \fi

    Since $(x_n)$ is bounded, by Bolzano-Weierstrass, it contains a convergent subsequence $a_n$ that converges to $a$.
    By definition of convergence of a sequence (2.2.3):
    \begin{align*}
        \forall \epsilon>0, \exists N\in \mathbf{N}: \forall n\geq N,  |a_n-a|<\epsilon.
    \end{align*}
    $(x_n)$ diverges, so does not satisfy to converge to $a$, then use the negation statement:
    \begin{align*}
        \exists \epsilon>0, \forall N \in \mathbf{N}:  \exists k \geq N, |x_k-a| \geq \epsilon.
    \end{align*}
    %here we use the fact that $\lnot (P\Rightarrow Q)$ is logically equivalent to $P\land (\lnot Q)$
    there should be infinite $k$ that satisfies: Suppose we choose $N_1 \in \mathbf{N}$ and $k_1 \in \mathbf{N}: k_1 \geq N_1, |x_{k_1}-a| \geq \epsilon$,
    then we can choose $N_2 = k_1$, there exists $k_2 \geq N_2 = k_1$, satisfies $|x_{k_2}-a|\geq \epsilon$, so we can continue and choose infinite $k$.\\
    Such $(x_k)$ is a subsequence of $(x_n)$, so is bounded. Again by B-W, it contains a convergent subsequence that converges to $b$.
    $b \neq a$ since all terms of $(x_k)$ are bounded away from $a$ by $\epsilon$. Thus, there are two subsequences that converge to different limits.
    \item If $(x_n)$ is monotone and contains a convergent subsequence, then $(x_n)$
    converges.\\
    \textbf{Solution} True. The subsequence is convergent, then it is bounded (by \textbf{Theorem 2.3.2}). We show 
    that a monotone sequence is bounded if it has a bounded subsequence.\\
    Without loss of generality, suppose $(x_n)$ is monotonically increasing. $(x_{n_k})$ is bounded by $\forall k\in \mathbf{N}, |x_{n_k}|\leq M$.
    Then, given any $n \in \mathbf{N}, \exists k \in \mathbf{N}: n \leq n_k, x_n \leq x_{n_k} \leq M$. Thus, $(x_n)$ is bounded. Similar for $(x_n)$ decreases.\\
    $(x_n)$ is bounded and monotone, then it converges (by \textbf{Theorem 2.4.2 MCT}).
\end{enumerate}

\subsection*{Exercise 2.5.9}
Let $(a_n)$ be a bounded sequence, and define the set
\begin{align*}
    S = \{x \in \mathbf{R}: x< a_n \text{~~for infinitely many terms~~} a_n\}.
\end{align*}
Show that there exists a subsequence $(a_{n_k} )$ converging to $s = \sup S$. (This is a
direct proof of the Bolzano-Weierstrass Theorem using the Axiom of
Completeness.)\\
\textbf{Solution} \textit{Proof} Since $s = \sup S$, then
\begin{align*}
    \forall \epsilon > 0, \exists x \in S: x+\epsilon>s
\end{align*}
Thus, $|x-s|< \epsilon$, i.e. for any $\epsilon >0$, 
there exists $N \in \mathbf{N}$ such that
\begin{align*}
    \forall n \geq N, s-\epsilon<a_n<s+\epsilon
\end{align*}
Given $\epsilon = \frac{1}{k}$,$\exists n_k\in \mathbf{N}, s-\frac{1}{k}<a_{n_k}<s+\frac{1}{k}$.\\
Therefore, $\lim a_{n_k} = s$.

\section{2.7 ~~Properties of Infinite Series}
\subsection*{Exercise 2.7.2}
Decide whether each of the following series converges or
diverges:\
\begin{enumerate}[label=(\alph*)]
    \item $\sum_{n=1}^{\infty} \frac{1}{2^n+n}$\\
    \textbf{Solution} Note that $0 \leq \frac{1}{2^n+1} \leq \frac{1}{2^n}$, and $\sum_{n=1}^{\infty}\frac{1}{2^n}$
    converges, since $|\frac{1}{2}|<1$. By \textbf{Theorem 2.7.4 (Comparison Test)}, $\sum_{n=1}^{\infty} \frac{1}{2^n+n}$ converges.
    \item $\sum_{n=1}^{\infty} \frac{\sin(n)}{n^2}$\\
    \textbf{Solution} Note that $0 \leq \frac{\sin(n)}{n^2} \leq \frac{1}{n^2}$, and $\sum_{n=1}^{\infty}\frac{1}{n^2}$
    converges, since the power $2>1$. By comperison test, it converges.
    \item $1-\frac{3}{4}+\frac{4}{6}-\frac{5}{8}+\frac{6}{10}-\frac{7}{12}+\cdots$\\
    \textbf{Solution} $a_n = (-1)^{n+1}(\frac{1}{2} + \frac{1}{2n}) = (-1)^{n+1}\frac{1}{2} + (-1)^{n+1}\frac{1}{2n}$. 
    where $(-1)^{n+1}\frac{1}{2}$ diverges apparently, and $(-1)^{n+1}\frac{1}{2n}$ converges by \textbf{Theorem 2.7.7 (Alternating Series Test)}. As a result, $(a_n)$ diverges.
    \item $1+\frac{1}{2}-\frac{1}{3}+\frac{1}{4}+\frac{1}{5}-\frac{1}{6}+\frac{1}{7}+\frac{1}{8}-\frac{1}{9}+\cdots$\\
    \textbf{Solution} $s_{3n}-s_{3(n-1)}=\frac{1}{3n-2}+\frac{1}{3n-1}-\frac{1}{3n} > \frac{1}{3(n-1)}$. We know that 
    $\sum_{n=2}^{\infty} \frac{1}{3(n-1)}$ diverges since the power $p=1$. Then by the comperison test, $\lim_{n} s_n = s_3+\sum_{n=2}^{\infty} [s_{3n}-s_{3(n-1)}]$ diverges.
    \item $1-\frac{1}{2^2}+\frac{1}{3}-\frac{1}{4^2}+\frac{1}{5}-\frac{1}{6^2}+\frac{1}{7}-\frac{1}{8^2}+\cdots$\\
    \textbf{Solution} $s_{2n}-s_{2(n-1)} = \frac{1}{2n-1}-\frac{1}{(2n)^2}$. Then $\lim_n s_n = s_2 + \sum_{n=2}^{\infty} [s_{2n}-s_{2(n-1)}] = \sum_{n=1}^{\infty} \frac{1}{2n-1} - \sum_{n=1}^{\infty}\frac{1}{(2n)^2}$.
    $\frac{1}{2n-1} > \frac{1}{2n}$, and $\sum_{n=1} \frac{1}{2n}$ diverges, by comperison test, $\sum_{n=1}^{\infty} \frac{1}{2n-1}$ diverges.\\
    $\sum_{n=1}^{\infty} \frac{1}{(2n)^2}$ converges cince the power $p=2>1$. As a result, $(s_n)$ diverges.
\end{enumerate}

\subsection*{Exercise 2.7.4}
Give an example of each or explain why the request is impossible
referencing the proper theorem(s).\
\begin{enumerate}[label=(\alph*)]
    \item Two series $\sum x_n$ and $\sum y_n$ that both diverge but where $\sum x_ny_n$ converges.\\
    \textbf{Solution} Let $x_n = \frac{1}{n}$ and $y_n=\frac{1}{n+1}$. Both series $\sum x_n$ and $\sum y_n$ 
    diverges because they're harmonic series with power 1. $\sum x_n y_n = \sum \frac{1}{n(n+1)}$ converges by 
    Comparison test with $\sum \frac{1}{n^2}$.
    \item A convergent series $\sum x_n$ and a bounded sequence $(y_n)$ such that $\sum x_ny_n$ diverges.\\
    \textbf{Solution} Let $x_n = (-1)^n \frac{1}{n}$ being convergent by the Alternating series test, and $y_n = (-1)^n$ is a bounded sequence. 
    $\sum x_n y_n = \frac{1}{n}$ diverges.
    \item Two sequences $(x_n)$ and $(y_n)$ where $\sum x_n$ and $\sum (x_n+y_n)$ both converge but $\sum y_n$ diverges.\\
    \textbf{Solution} Impossible. Since if $\sum x_n$ and $\sum (x_n+y_n)$ both converge, by \textbf{Theorem 2.7.1 (Algebraic Limit Theorem for Series)}, 
    $\sum y_n$ should also be convergent.
    \item A sequence $(x_n)$ satisfying $0 \leq x_n \leq 1/n$ where $\sum(-1)^n x_n$ diverges.\\
    \textbf{Solution} Let $$x_n=
    \begin{cases}
        \frac{1}{n} & \text{n is odd}\\
        0 & \text{n is even}
    \end{cases}$$
    It satifies $0 \leq x_n \leq \frac{1}{n}$ obviously and $\sum(-1)^n x_n$ diverges because it's harmonic series with power 1.
\end{enumerate}

\chapter{Basic Topology of $\mathbf{R}$}
\section{3.2 ~~Open and Closed Sets}
\subsection*{Exercise 3.2.2}
\begin{align*}
    A = \Bigl\{(-1)^n +\frac{2}{n}:n= 1,2,3,\ldots\Bigr\} \text{~~~~and~~~~} B=\{x\in \mathbf{Q}:0<x<1\}
\end{align*}
Answer the following questions for each set:
\begin{enumerate}[label=(\alph*)]
    \item What are the limit points?\\
    \sol The set of limit points of $A$ is $\{-1,1\}$. The set of limit points of $B$ is $[0,1]$.
    \item Is the set open? Closed?\\
    \sol $A$ is not open since $1\in A$ does not have an open interval $(a,b)\in A$, and not closed since $-1 \notin A$. 
    $B$ is not open since $\forall x \in B, \nexists (a,b) : a<x<b \land (a,b)\in B$, and not closed since $\exists^{\infty} x \in [0,1] \land x \in \mathbf{Q}$.
    \item Does the set contain any isolated points?\\
    \sol Every points in $A$ except $1$ are isolated points. $B$ has no isolated points.
    \item Find the closure of the set.\\
    \sol $\bar{A}=A \cup \{-1\}$ and $\bar{B}= B \cup [0,1] = [0,1]$.
\end{enumerate}
\subsection*{Exercise 3.2.5}
Prove \textbf{Theorem 3.2.8.} \textit{A set $F \subseteq \mathbf{R}$ is closed if and only if every Cauchy sequence contained 
in $F$ has a limit that is also an element of $F$.}\\
\sol \textit{Proof} \\
$\Rightarrow$ Suppose set $F \subseteq \mathbf{R}$ is closed, and let $(x_n)$ be a Cauchy sequence contained in $F$. 
By Cauchy Criterion, $(x_n)\to x$, $x$ is then a limit point of $F$. Thus, $x \in F$.\\
$\Leftarrow$ Suppose every Cauchy sequence in $F$ converges to a limit in $F$. Let $x$ be a limit point of $F$. 
By \textbf{Theorem 3.2.5}, $\exists (a_n) \in F: x=\lim a_n$, and by Cauchy Criterion, $(a_n)$ must be a Cauchy sequence, 
then $x \in F$, and $F$ is closed. \qed
\subsection*{Exercise 3.2.11}
\begin{enumerate}[label=(\alph*)]
    \item Prove that $\overline{A \cup B}=\overline{A}\cup\overline{B}$.\\
    \sol \textit{Proof} Given set $A,B$ and let $A', B'$ be the set of all limit points of $A$ and $B$, respectively. 
    And $(A \cup B)'$ be the set of all limit points of $A \cup B$.\\
    $\subseteq$ If $x \in \overline{A \cup B}$, $x \in (A \cup B)\cup(A \cup B)'$, i.e. 
    $(x \in A) \lor (x\in B) \lor (x\in (A \cup B)')$. If $(x \in A) \lor (x \in B)$, then obviously 
    $x \in \overline{A}\cup\overline{B}$. If $x\in (A \cup B)'$, $x =\lim (x_n)$ for some sequence $x_n$ 
    contained in $A\cup B$, then $((x_n) \in A) \lor ((x_n) \in B)$. If $(x_n) \in A$, then $x \in A'$, 
    similar for $(x_n) \in B$, thus $x \in (A'\cup B')$, $x \in \overline{A}\cup\overline{B}$.\\
    $\supseteq$ If $x \in \overline{A}\cup\overline{B}$, $(x \in (A \cup A')) \lor (x \in (B \cup B'))$, i.e. 
    $(x \in A) \lor (x \in A') \lor (x \in B) \lor (x\in B')$. Similarly, we consider $(x \in A') \lor (x \in B')$, then 
    $x =\lim x_n$ for some sequence $((x_n)\in A) \lor ((x_n) \in B)$, then $(x_n) \in (A \cup B)$, thus 
    $x \in (A \cup B)', x\in \overline{A\cup B}$.\\
    That completes the proof. \qed  
    \item Does this result about closures extend to infinite union of sets?\\
    \sol No, it does not. Because the previous proof relies on the fact that 
    the closure is a closed set and the union of a finite collection of closed sets is closed. 
    While infinite union of sets are not always closed. For example, $A_n = \{\frac{1}{n}\}$ is a closed set for each $n \in \mathbf{N}$, 
    and $\bigcup_{n=1}^\infty A_n= \{\frac{1}{n}: n \in \mathbf{N}\}$, which is a open set with limit point 0.
    \begin{align*}
        \overline{\bigcup_{n=1}^\infty A_n}= \bigcup_{n=1}^\infty \overline{A_n} \cup \{0\}
    \end{align*}
\end{enumerate}

\section{3.3 ~~Compact Sets}
\subsection*{Exercise 3.3.4}
Assume $K$ is compact and $F$ is closed. Decide if the following
sets are definitely compact, definitely closed, both, or neither.
\begin{enumerate}[label=(\alph*)]
    \item $K \cap F$\\
    \sol $K$ is compact so is closed and bounded, $F$ is closed. $K \cap F$ is the intersection of closed sets, is closed. 
    $K \cap F$ is a subset of $K$, thus bounded. So $K \cap F$ is compact.
    \item $\overline{F^c \cup K^c}$\\
    \sol The form of closure implies closed. $K$ is bounded implies $K^c$ is unbounded, so is 
    $F^c \cup K^c$ and $\overline{F^c \cup K^c}$.
    \item $K\backslash F = \{x \in K: x\notin F\}$\\
    \sol $K\backslash F = K \cap F^c$, which is a subset of $K$, thus bounded. While it can be closed or not. 
    Let $K'$ be the set of limit points of $K$. If $K \subseteq F^c $, then $K\backslash F = K$, is still closed. 
    If $K \subseteq F^c$ and $K' \cap F \neq \emptyset$, i.e. there is limit points of $K$ in set $F$, 
    then it is also a limit point of $K\backslash F$ but not in $K\backslash F$, thus $K\backslash F$ is not closed.
    \item $\overline{K \cap F^c}$\\
    \sol The form of closure implies being closed. From (c) we know that $K \cap F^c$ is bounded, so is $\overline{K \cap F^c}$. 
    Thus, $\overline{K \cap F^c}$ is compact.
\end{enumerate}
\subsection*{Exercise 3.3.8}
Let $K$ and $L$ be nonempty compact sets, and define
\begin{align*}
    d=\inf\{|x-y|:x \in K \text{~~and~~} y \in L\}.
\end{align*}
This turns out to be a reasonable definition for the \textit{distance} between $K$ and $L$.
\begin{enumerate}[label=(\alph*)]
    \item If $K$ and $L$ are disjoint, show $d >0$ and that $d=|x_0-y_0|$ for some $x_0 \in K$ and 
    $y_0 \in L$.\\
    \sol $K$ and $L$ being compact implies $\{|x-y|:x \in K \text{~~and~~} y \in L\}$ is a compact set. 
    Since $K$ and $L$ are disjoint, $\nexists x \in K \land x \in L$, $|x-y| >0, d >0$.
    $\{|x-y|:x \in K \text{~~and~~} y \in L\}$ is closed, the limit point $d=|x_0-y_0|$ is in the set. 
    Thus, $d=|x_0-y_0|$ for some $x_0 \in K$ and $y_0 \in L$.
    \item Show that it's possible to have $d=0$ if we assume only that the disjoint sets $K$ 
    and $L$ are closed.\\
    \sol Let $K = \mathbf{R}$ and $L=\mathbf{N}$, both are closed and unbounded. Then $d=0$.
\end{enumerate}

\chapter{Functional Limits and Continuity}
\section{4.2 ~~Functional Limits}
\subsection*{Exercise 4.2.5}
Use Definition 4.2.1 to supply a proper proof for the following
limit statements.
\begin{enumerate}[label=(\alph*)]
    \item $\lim_{x\to2}(3x+4)=10.$\\
    \sol $|(3x+4)-10|=|3x-6|=3|x-2|$. Given $\epsilon >0$, choose $\delta = \epsilon/3$, 
    then $0<|x-2|<\delta$ implies $|(3x+4)-10|<3\delta=3(\epsilon/3)=\epsilon$.
    \item $\lim_{x\to0}x^3=0.$\\
    \sol $|x^3-0|=|x^3|$. Given $\epsilon >0$, choose $\delta = \epsilon^{\frac{1}{3}}$, 
    then $0<|x-0|<\delta$ implies $|x^3-0|<\delta^3=\epsilon$.
    \item $\lim_{x\to2}(x^2+x-1)=5.$\\
    \sol $|(x^2+x-1)-5|=|x^2+x-6|=|x-2||x+3|$. Given $\epsilon >0$, choose $\delta = \min \{1,\epsilon/6\}$. 
    If $0<|x-2|<\delta$, then 
    \begin{align*}
        |(x^2+x-1)-5|=|x-2||x+3|<\bigg(\frac{\epsilon}{6}\bigg)6=\epsilon.
    \end{align*}
    \item $\lim_{x\to3}\frac{1}{x}=\frac{1}{3}.$\\
    \sol $\bigg|\frac{1}{x}-\frac{1}{3}\bigg|=\frac{|x-3|}{3|x|}$. Given $\epsilon >0$, 
    choose $\delta=\min\{1,6\epsilon\}$. If $0<|x-3|<\delta$, then
    \begin{align*}
        \bigg|\frac{1}{x}-\frac{1}{3}\bigg|=\frac{|x-3|}{3|x|}<6\epsilon\bigg(\frac{1}{6}\bigg)=\epsilon.
    \end{align*}
\end{enumerate}
\subsection*{Exercise 4.2.7}
Let $g : A \to \mathbf{R}$ and assume that $f$ is a bounded function on $A$
in the sense that there exists $M >0$ satisfying $|f(x)| \leq M$ for all $x \in A$.\\
Show that if $\lim_{x\to c} g(x) = 0$, then $\lim_{x\to c} g(x)f(x) = 0$ as well.\\
\sol \\\textit{Proof} If $\lim_{x\to c} g(x) = 0$, by definition:
\begin{align*}
    \forall \epsilon_1>0, \exists \delta>0 : 0<|x-c|<\delta \Rightarrow |g(x)|<\epsilon_1.
\end{align*}
and we know that
\begin{align*}
    \exists M>0, \forall x \in A: |f(x)|\leq M.
\end{align*}
Given $\epsilon >0$, let $\epsilon_1=\frac{\epsilon}{M}$, then there exists $\delta$ such that 
if $0<|x-c|<\delta$, $|g(x)|<\epsilon_1=\frac{\epsilon}{M}$. Thus,
\begin{align*}
    |g(x)f(x)-0|=|g(x)||f(x)|<\biggl(\frac{\epsilon}{M}\biggr)M=\epsilon.
\end{align*}
\qed
\end{document}